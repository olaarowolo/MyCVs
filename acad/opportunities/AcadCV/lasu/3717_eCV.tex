\documentclass[11pt,a4paper]{article}

\usepackage[margin=1in]{geometry}
\usepackage[T1]{fontenc}
\usepackage[utf8]{inputenc}
\usepackage{enumitem}
\usepackage{hyperref}
\usepackage{parskip}

\setlist[itemize]{leftmargin=*,label=--}
\setlist[enumerate]{leftmargin=*}

\begin{document}

\begin{center}
\textbf{LAGOS STATE UNIVERSITY, OJO} \\
LASU Academic Staff Annual Performance Evaluation System \\
\textbf{ACADEMIC STAFF CURRICULUM VITAE}
\end{center}

Printed: Today, 08 Feb 2026, 01:05pm

\section*{Staff Biodata}
\begin{itemize}
  \item Name: AROWOLO Sunday Olasunkanmi
  \item Personal File Number: 3717
  \item Post Applied For: Assistant Lecturer
  \item Faculty/Department: School of Communications / Journalism
  \item Present Position: Assistant Lecturer
  \item Place and Date of Birth: 17/01/1993
  \item Gender: MALE
  \item Nationality: Nigerian
  \item Marital Status: Married
  \item State of Origin / Local Govt Area: Lagos / Ojo
  \item Number and Ages of Children: Number (0); Ages (NIL)
\end{itemize}

\section*{Contact}
\begin{itemize}
  \item Residential Address: 4, AROWOLO LANE, AROWOLO BUS-STOP, IBA TOWN, OJO, LAGOS, NIGERIA
  \item Phone Number(s): 08062081819
  \item Email Address: \href{mailto:olasunkanmi.arowolo@lasu.edu.ng}{olasunkanmi.arowolo@lasu.edu.ng}
  \item Correspondence: DEPARTMENT OF JOURNALISM, SCHOOL OF COMMUNICATION, LAGOS STATE UNIVERSITY, OJO, LAGOS, NIGERIA
\end{itemize}

\section*{Educational Institutions Attended With Dates}
\begin{itemize}
  \item 2021 -- 2025: University of Kent, England, United Kingdom
  \item 2016 -- 2019: Department of Journalism, School of Communication, Lagos State University, Ojo, Lagos, Nigeria
  \item 2010 -- 2015: Lagos State University, Ojo
  \item 2007 -- 2009: Henna-Teck International College, Iba Town
  \item 2003 -- 2007: Pakun-Ola Comprehensive College, Iba New Town
  \item 1996 -- 2003: Command Childrens School, Signal Barracks, Mile 2
\end{itemize}

\section*{Academic Qualifications With Dates}
\begin{itemize}
  \item * 2025: PhD in Journalism (no corrections)
  \item 2019: M.Sc. in Communication Studies
  \item 2015: B.Sc. in Mass Communication
  \item 2009: Senior School Certificate (SSCE)
  \item 2003: Primary School Leaving Certificate (Nursery/Primary)
\end{itemize}

\section*{Professional Qualifications / Certifications With Dates}
\begin{itemize}
  \item 2020: International Computer Driving License (ICDL), awarded by ICDL, sponsored by TetFund in Nigeria
\end{itemize}

\section*{Training Programmes Attended With Dates}
\begin{itemize}
  \item 24/08/2020 -- 28/08/2020: International Computer Driving License (ICDL), organized by ICDL, sponsored by TetFund in Nigeria
  \item 09/10/2015 -- 09/10/2015: One-day refresher workshop on how to report and connect with the people, organized by Journalism Clinic
  \item 27/03/2014 -- 21/08/2014: Google Map Maker, organized by Google Inc.
  \item 25/06/2014 -- 27/06/2014: The Business of Outdoor Advertising in Africa: Challenges and Opportunities, organized by Lagos State Signage \& Advertising Agency (LASAA)
  \item 13/03/2013 -- 15/03/2013: Negotiation and Conflict Management, organized by United States Institute of Peace
  \item 18/05/2018 -- 20/05/2018: The New Media: Responsibilities, Opportunities and Challenges (ROC) for the Campus Journalist, organized by The Nation Campus Life Newspaper
  \item Graduate Researcher College, University of Kent, England, United Kingdom (57 sessions; selected highlights)
  \begin{itemize}
    \item 21/05/2025: Milestone 4: Finishing your PhD, Online\\ Summary/Impact (transferable skill): Final-stage planning and completion strategies
    \item 04/09/2024: Milestone 5: Preparing for your Viva, In person (Canterbury)\\ Summary/Impact (transferable skill): Viva preparation and examiner expectations
    \item 13/06/2024: Advanced Skills in NVivo, Online\\ Summary/Impact (transferable skill): Advanced qualitative analysis techniques
    \item 20/04/2023: Navigating the Peer Review Process, Online\\ Summary/Impact (transferable skill): Responding to peer review and revisions
    \item 22/03/2023: Research Data Management Planning, In person (Canterbury)\\ Summary/Impact (transferable skill): Data planning, governance, and sharing
    \item 01/03/2023: Preparing for Fieldwork, In person (Canterbury)\\ Summary/Impact (transferable skill): Fieldwork design, ethics, and logistics
    \item 31/01/2023: Writing Articles for International Publication in Peer-Reviewed Journals (Humanities \& Social Sciences), In person (Canterbury)\\ Summary/Impact (transferable skill): Publishing strategy and manuscript positioning
    \item 03/11/2022: Introductory Statistics, In person (Canterbury)\\ Summary/Impact (transferable skill): Quantitative analysis foundations
    \item 20/10/2022: Academic CVs, In person (Canterbury)\\ Summary/Impact (transferable skill): Academic CV structure and positioning
    \item 25/02/2022: Producing a Research Poster, In person (Canterbury)\\ Summary/Impact (transferable skill): Research communication and visual design
    \item 01/11/2021: Developing as a Teacher, Online\\ Summary/Impact (transferable skill): Teaching practice and pedagogy development
    \item 26/10/2021: Leadership - Motivating Myself, Motivating Others, Online\\ Summary/Impact (transferable skill): Leadership and self-management skills
  \end{itemize}
  \item Impact statement: The training portfolio strengthened research design, academic writing, and professional communication, enhancing publication readiness and wider engagement.
\end{itemize}

\section*{Honours and Distinctions}
\begin{itemize}
  \item 2025: PhD (Awarded with no corrections)
  \item * 2025: Global Talent Endorsement, British Academy
  \item * 2021 -- 2025: Tertiary Education Trust Fund (TETFund) Federal Government of Nigeria University Doctoral Scholarship
  \item * 2024: 3-Minutes Thesis (Runner-up), University of Kent
\end{itemize}

\section*{Membership of Professional Bodies}
\begin{itemize}
  \item 2014: Member (Full), Association of Communication Scholars \& Professionals of Nigeria (ACSPN)
  \item * Member, International Communication Association (ICA)
  \item * Member, Public Communication of Science and Technology (PCST)
  \item * Member, African Council for Communication Education (ACCE)
\end{itemize}

\section*{Work Experience With Dates}
\textbf{Work Experience in the University}
\begin{itemize}
  \item * November 2023 -- December 2024: Lecturer (Media/Core Communication), Health and Social Care, East Kent College, Ashford (UK)
  \item * September 2022 -- 2023: Assistant Lecturer, Centre for Journalism, University of Kent, Rochester (UK)
  \item * September 2021 -- 2022: Assistant Lecturer (HPL), School of Social Policy, Sociology and Social Research, University of Kent (UK)
  \item 2019 -- till date: Journalism, Lagos State University -- MAC 204 (B.Sc.), Designation: Graduate Assistant
  \item 2019 -- till date: Journalism, Lagos State University -- MAC 208 (B.Sc.), Designation: Graduate Assistant
  \item 2019 -- till date: Journalism, Lagos State University -- MAC 322 (B.Sc.), Designation: Graduate Assistant
  \item 2019 -- till date: Journalism, Lagos State University -- MAC 308 (B.Sc.), Designation: Graduate Assistant
  \item 2017 -- 2019: Journalism, School of Communication, Lagos State University -- MAC (B.Sc.), Designation: Graduate Assistant
  \item 2015 -- 2016: Department of Languages, Linguistics, and Communication Studies, Osun State University -- CMS (B.Sc.), Designation: Graduate Assistant
\end{itemize}

\textbf{Work Experience in Other Organizations}
\begin{itemize}
  \item 2018 -- till date: Editor, Faculty of Science, LASU
  \begin{itemize}
    \item Gathering and production of Faculty of Science quarterly newsletter (eVersion and prints)
    \item Layout and design, editing of contents, creation of eVersion, and printing of hardcopy
  \end{itemize}
  \item 2016 -- 2018: Consultant, Academic Staff Cooperative, Lagos State University, Ojo
  \begin{itemize}
    \item Generate PR content on ASCOOP activities for monthly newsletter (eVersion and prints)
    \item Layout and design, editing of contents, creation of eVersion, and printing of hardcopy
  \end{itemize}
  \item 2013 -- 2013: Editor in Chief, LASUGONG Newspaper
  \begin{itemize}
    \item Oversees the newspaper publication for the semester
  \end{itemize}
  \item 2013 -- 2014: Intern, Sunday Vanguard, Vanguard Newspapers
  \begin{itemize}
    \item Special interpretive and investigative reports
  \end{itemize}
  \item 2012 -- 2013: Page Planner / Graphic Editor, Ehingbeti Media (Publisher of Eko Akete News)
  \begin{itemize}
    \item Plans pages and designs graphics
  \end{itemize}
  \item 2012 -- 2014: Book Layout Planning / Graphic Editor, Griot Publishers
  \begin{itemize}
    \item Book layout planning and graphic designs
  \end{itemize}
  \item 2012 -- 2013: Intern, Tertiary Education Beat, Sudan Media Total Education Newspaper
  \begin{itemize}
    \item Specialized reporting
  \end{itemize}
  \item 2011 -- 2014: Freelance Campus Journalist, National Newspapers
  \begin{itemize}
    \item Reporting on campus and external activities for The Nation, The Vanguard, Nigerian Tribune, The Sun, Nigerian Compass, and Campus Delight (online editor)
  \end{itemize}
\end{itemize}

\section*{Administrative and Leadership Experience in the University System With Dates}
\begin{itemize}
  \item January/2018 -- till date: Member, School of Communication Faculty Students Help-desk Committee, Lagos State University
  \item August/2018 -- October/2018: Member, Faculty Hand-Book Drafting Committee, School of Communication, Lagos State University
  \item March/2018 -- till date: Member, School of Communication Publication Committee, Lagos State University
  \item March/2018 -- till date: Business Manager, School of Communication / Faculty Journal Media and Communication Review, Lagos State University
  \item October/2017 -- till date: Member / Graphics Editor, LASUGONG Newspaper Publication Committee, Lagos State University
  \item * 2022 -- Present: Contributed to planning of teaching programmes, Centre for Journalism, University of Kent
  \item * 2022 -- Present: Collaborated on course development and curriculum changes, Centre for Journalism, University of Kent
\end{itemize}

\section*{Listing of Academic Publications}
\textbf{Published Chapters}
\begin{enumerate}
  \item Oso, Lai, Adeniran, R. and Arowolo, O., 2019, ``The Challenges and Implications of Health Journalism Practice in Nigerian Newspapers'', 115 -- 139 in Mass Media in Nigeria: Research, Theories and Practice, edited by Adeyanju, A.M., Jimoh, I. and Suleiman, H.M., Jos: The National Institute, Kuru, Nigeria
  \item Oso, L. and Arowolo, S. O., 2019, ``Commercialisation and tabliodisation: Journalism as infotainment.'', 189 -- 205 in Beyond Fun: Media Entertainment, Politics and Development in Nigeria, edited by L. Oso, R. Olatunji, O. Omojola and S. Oyero, Lagos: Malthouse Press Limited
  \item Oso, Lai and Arowolo, Sunday Olasunkanmi, 2018, ``In Search of an Elusive Ideal: The Socio-political Context of Journalistic Objectivity in Nigeria'', 75 -- 98 in Mass Communication Education and Practice in Nigeria, edited by Ozohu-Suleiman, Y. and Muhammed, S., Zaria: Ahmadu Bello University Press Ltd
  \item * Arowolo, S.O. and Animashaun, O.W., 2020, ``Text Language: A Revolution in Modern Communication'', in Book in Honour Professor Lai Oso at 60, Research and Book Publication Committee, School of Communication, Lagos State University, Lagos, Nigeria
\end{enumerate}

\textbf{Journal Articles}
\begin{enumerate}
  \item * Shodipe, O.A., Arowolo, O., Oloyede, I.B., Alade, M., Fadeyi, I.O. and Oko-Epelle, L., 2024, ``Visual portrayal of Monkeypox outbreak on BBC: is Western media biased against Africa?'', Global Knowledge, Memory and Communication (ahead-of-print)
  \item * Oso, Lai, Adeniran, R. and Arowolo, O., 2024, ``Journalism Ethics: The Dilemma, Social and Contextual Constraints'', Cogent Social Sciences
  \item * Arowolo, O., Yusuf, K., Yahya, J., Lambo, O. and Oduolowu, D., 2023, ``World Bank Concept of Good Governance, Contemporary Nigerian Democracy, and the Place of the Media'', East African Journal of Interdisciplinary Studies, 6(1), pp. 54--69
  \item * Suleiman, H., Adedeji, K. and Arowolo, S. O., 2020, ``Online Newspaper Coverage of the 8th National Assembly in Nigeria'', Media and Communication Review (MCR), Vol. 2, Issue 2, pp. 57--83
  \item Asomba, I.T. and Arowolo, S.O., 2018, ``Politics of Nigerian Newspaper Coverage of Ondo State 2012 Gubernatorial Election.'', Media and Communication Review (MCR), Vol. 3, Issue 1, pp. 85 -- 109
  \item Arowolo, S. O., 2015, ``Uses and Gratification theory and Tobacco Advertising necessity: An Expos'', International Journal of Broadcasting and Information Technology (IJBCT), Vol. 3, Issue 1, pp. 81 -- 98
\end{enumerate}

\section*{Area of Specialization}
Journalism and Media Communication (Political Communication, Development Communication, Digital Media)

\section*{Research Interest}
\begin{itemize}
  \item Journalism
  \item Development Communication
  \item * Independent journalism and comparative media systems
  \item * Media and democratic processes in developing nations and the UK
  \item * Framing analysis and media representation of marginalized voices
\end{itemize}

\section*{Current Research (Detailed)}
\begin{enumerate}
  \item \textbf{Topic: Comparative study of independent journalism funding models in the UK and Nigeria}\\
  \textbf{Detail of Research}: Introduction: This study examines how independent news organizations sustain editorial autonomy amid digital disruption, declining advertising, and policy gaps. It builds on ongoing work on media resilience and entrepreneurial journalism in Nigeria, and extends it to a UK--Nigeria comparative perspective. Aims/Objectives: Map newsroom funding ecosystems, assess how funding shapes independence, and identify viable, resilient models that balance public interest with sustainability. Methodology: Mixed methods combining content analysis of outlet practices, surveys and interviews with editors, publishers, and funders, focus groups with media entrepreneurs, and policy document analysis. Comparative case selection is guided by market size, ownership type, and regulatory context. Expected Result: A typology of funding models, evidence of independence risks and mitigation strategies, and a set of policy and industry recommendations. Contribution to knowledge/society: Provides a comparative framework for sustainable journalism, supports media entrepreneurship, and informs public policy on democratic accountability and media pluralism.

  \item \textbf{Topic: Reclaiming authenticity in Nigerian press coverage of the National Social Investment Programme (NSIP)}\\
  \textbf{Detail of Research}: Introduction: NSIP coverage is often shaped by press releases and elite sources, limiting authentic representation of beneficiaries and policy outcomes. This project investigates how authenticity can be recovered in reportage of poverty alleviation programmes. Aims/Objectives: Identify source dominance patterns, assess authenticity indicators in narratives, and propose editorial practices that strengthen public trust and accountability. Methodology: Content analysis and critical discourse analysis of Nigerian newspaper coverage of NSIP across 2016--2025, combined with interviews with journalists to understand newsroom routines and verification constraints. The study uses a structured coding scheme for sources, frames, and beneficiary visibility. Expected Result: Evidence of how source dependency and institutional pressures affect authenticity, alongside a practical framework for reporting standards that elevate beneficiary voices. Contribution to knowledge/society: Advances development communication scholarship, improves the quality of policy reporting, and supports more citizen-centered media practices.

  \item \textbf{Topic: The side-lined voice of the poor: slants and framing of poverty by the Nigerian print media}\\
  \textbf{Detail of Research}: Introduction: Media framing shapes public understanding of poverty, yet the lived experiences of poor communities are often marginalized in coverage of welfare and social investment. This study focuses on how Nigerian print media construct poverty narratives and whose voices are amplified or ignored. Aims/Objectives: Examine dominant frames, identify slants across outlets, and measure the visibility of beneficiary perspectives in poverty reporting. Methodology: The project applies framing theory to a corpus of NSIP-related newspaper coverage, using thematic coding for positive and negative frames, and source analysis to assess elite vs. citizen representation. It also draws on journalist interviews to explain newsroom decisions and gatekeeping. Expected Result: A map of framing patterns and evidence of systemic marginalization of poor voices, with cross-newspaper comparisons of editorial bias. Contribution to knowledge/society: Provides empirical grounding for inclusive reporting standards, strengthens accountability journalism, and deepens understanding of poverty communication in African media systems.

  \item \textbf{Topic: Voices of the oppressed: Nigeria Twittersphere conversations on the Tradermoni poverty programme}\\
  \textbf{Detail of Research}: Introduction: Social media provides alternative public spheres for citizens to contest official narratives of welfare programmes. This project studies discourse on X (formerly Twitter) about Tradermoni, a micro-credit component of NSIP, particularly during politically sensitive periods. Aims/Objectives: Identify dominant themes in citizen discourse, assess how online narratives challenge or reinforce mainstream media frames, and evaluate the role of digital platforms in accountability. Methodology: Critical discourse analysis of public posts and hashtags related to Tradermoni, triangulated with newspaper coverage to detect convergence or divergence in framing. The study uses purposive sampling around key policy announcements and election cycles. Expected Result: A typology of citizen narratives, evidence of counter-framing against official claims, and insights into digital skepticism or support. Contribution to knowledge/society: Enhances understanding of digital public spheres in Nigeria, informs policy communication strategies, and highlights how marginalized voices participate in welfare debates.

  \item \textbf{Topic: What\'s up? News trading in the Nigerian press}\\
  \textbf{Detail of Research}: Introduction: News trading and content commodification shape newsroom routines, raising concerns about independence, verification, and public interest reporting. This project explores how commercial pressures influence news selection and sourcing in Nigerian newspapers. Aims/Objectives: Document practices of news trading and information subsidies, assess their impact on editorial autonomy, and identify institutional constraints that incentivize shallow or recycled coverage. Methodology: Content analysis of newspaper stories with attention to source patterns and press release reliance, complemented by interviews with journalists and editors about commercial and organizational pressures. The study situates findings within political economy and professional journalism frameworks. Expected Result: Empirical evidence of trading practices, indicators of churnalism, and recommendations for editorial safeguards that protect investigative capacity. Contribution to knowledge/society: Clarifies the relationship between market pressures and journalistic quality, supports reforms in newsroom practice, and strengthens public trust in the press.

  \item \textbf{Topic: Bottom-up news: do ordinary people matter in the news? -- a study of Tradermoni coverage in Nigeria}\\
  \textbf{Detail of Research}: Introduction: Tradermoni coverage provides a strong case for testing whether Nigerian newspapers prioritize ordinary people or elite voices in development reporting. This project interrogates bottom-up vs. top-down narratives in coverage of micro-credit distribution. Aims/Objectives: Measure the visibility of beneficiary voices, compare narrative structures across newspapers, and assess how political alignment shapes source selection. Methodology: Quantitative content analysis and qualitative discourse analysis of Tradermoni stories in four major newspapers, with systematic coding of sources, frames, and narrative emphasis. The design draws on social construction of news theory and elite source dominance literature. Expected Result: Clear evidence of how frequently ordinary people appear in coverage and how election periods intensify elite-driven narratives. Contribution to knowledge/society: Provides practical guidance for participatory journalism, strengthens accountability in welfare reporting, and offers a Nigerian case study for bottom-up news theory.

  \item \textbf{Topic: Echoing the frames: Nigerian print media accounts of the Lekki Toll Gate Massacre}\\
  \textbf{Detail of Research}: Introduction: The Lekki Toll Gate events are a defining moment in contemporary Nigerian civic life, yet media framing varies by outlet and political context. This study examines how print media constructed narratives of responsibility, victimhood, and state legitimacy. Aims/Objectives: Identify dominant frames, analyze source diversity, and assess how newspapers balanced eyewitness accounts, official statements, and human rights perspectives. Methodology: Frame analysis of newspaper reports during and after the event, supported by source coding and comparative editorial analysis across national outlets. The project employs critical discourse analysis to interpret language choices and narrative positioning. Expected Result: A detailed account of competing frames and the extent to which citizen testimonies were centered or marginalized. Contribution to knowledge/society: Contributes to scholarship on media, protest, and state power, and provides evidence for ethical reporting standards in crisis and human rights contexts.

  \item \textbf{Topic: AI applications in journalism research and practice}\\
  \textbf{Detail of Research}: Introduction: AI is reshaping journalistic workflows through automated analysis, content generation, and bias detection. This project examines how AI can be used responsibly in journalism research and newsroom practice, grounded in evidence from media coverage studies. Aims/Objectives: Evaluate practical AI use cases, assess ethical risks, and develop guidance for integrating AI into research and reporting without undermining editorial judgment. Methodology: A combination of conceptual analysis, review of AI-assisted journalism tools, and pilot computational text analysis applied to media datasets. The project draws on empirical insights from NSIP coverage to identify tasks where AI can improve verification, sentiment analysis, and source diversity checks. Expected Result: A framework for AI-assisted journalism that balances efficiency with accountability, plus recommended safeguards for human oversight. Contribution to knowledge/society: Advances methodological innovation in media research, supports ethical newsroom transformation, and informs training for journalists in data-driven reporting.

  \item \textbf{Topic: Algorithmic bias in media coverage and digital platform governance}\\
  \textbf{Detail of Research}: Introduction: Platform algorithms shape what audiences see, yet bias in ranking, moderation, and recommendation systems can distort public discourse. This project investigates how algorithmic bias affects media visibility and how governance frameworks can improve transparency and accountability. Aims/Objectives: Identify mechanisms of bias, examine governance gaps in platform policy, and propose principles for fairer media distribution. Methodology: Policy and literature analysis combined with case-based evaluation of platform practices and bias risks. The study integrates insights from digital discourse analysis and ethical journalism theory to assess how algorithmic choices influence coverage of social policy issues. Expected Result: A set of governance recommendations, including audit practices and transparency standards for platforms and publishers. Contribution to knowledge/society: Supports democratic communication by clarifying how platform power shapes news exposure, and offers evidence-based guidance for regulators, media organizations, and civil society.
\end{enumerate}

\section*{Research Collaboration Experience}
2018: Association of Communication Scholars and Professionals of Nigeria (ACSPN) -- Understanding Nigerian Media and Elections through research: Analysis of the 2015 presidential election campaign messages

\section*{Conferences / Workshops Attended and Papers Presented With Dates}
\begin{itemize}
  \item * 2025: Arowolo, S. Olasunkanmi and Oduolowu, D., ``Communicating Climate Change in Africa: Role perception and role shifting among environmental journalists in Nigeria'', PCST 2025
  \item * 2024: Arowolo, S. Olasunkanmi, ``Harnessing the Fourth Estate: News media as catalyst for effective social investment policies implementation'', AEJMC Midwinter Conference (Political Communication Division), University of Oklahoma
  \item * 2024: Arowolo, S. Olasunkanmi and Oduolowu, D., ``Cross-Border Perspectives: A Comparative Thematic Analysis of Local Governance Discourse in US and UK Online Community News'', AEJMC Southeast Colloquium, University of Kentucky
  \item 2018: Oso, Lai and Arowolo, S. Olasunkanmi, ``Hate Speech and Negotiating'', 5th ACSPN Annual Conference on Hate Speech, Fake News and Political Stability in Africa, September 5--6, 2018, Asaba, Nigeria
  \item 2018: Arowolo, S. Olasunkanmi and Akano, M. M., ``New media and attitude of students towards online shopping: A case of Jumia.ng'', ABUAD Media, Communication, Information Technology (MCIT) 2018 International Conference, September 19--22, 2018, Ado-Ekiti, Ekiti State
  \item 2018: Biobaku, M.O.; Arowolo, S.O.; Odesanya, L.A.; and Fatonji, S., ``Influence of The Lion of Bourdillon Documentary on the Voting Intentions of Lagos Electorates in Nigeria's 2015 General Elections'' (Press Freedom Day Conference)
  \item 2018: Ogundoyin, S.O. and Arowolo, S.O., ``Media Advocacy Theory: A content analysis of media coverage of Nigerians migration to Europe via Libya'' (ACCE National Conference)
  \item 2017: Oso, L.; Jimoh, J.; and Arowolo, O.S., ``3Ws: The social composition of news in selected Nigerian newspapers'', Sept. 5--8, 2017, Faculty of Communication, Bayero University, Kano (BUK)
  \item 2016: Suleiman, H.B.; Arowolo, O.S.; and Olona, K.A., ``Legislative Reporting in Nigeria: Online Newspaper 100 days Coverage of the 8th National Assembly Resolution'', Ebenezer Soola Conference on Communication, Sept. 27--30, 2016, University of Ibadan, Ibadan, Nigeria
\end{itemize}

\section*{Extra-Curricular Activities}
\begin{itemize}
  \item Information and Communication Technology Officer II, Association of Communication Scholars \& Professionals of Nigeria (ACSPN)
  \item Consultancy in Mass Media Graphic Designs and Publishing
  \item * Founder and CEO, Afriscribe (afriscribe.org)
  \item * Director and Founder, ResearchAfrica (researchafrica.pub)
\end{itemize}

\section*{Scholarship and Prizes (University, Secondary, or Technical Level)}
\begin{itemize}
  \item 2013 -- 2014: Scholarship -- Lagos State Local Scholarship
\end{itemize}

\section*{Referees}
\begin{itemize}
  \item * 1. Dr Ben Cocking
  \begin{itemize}
    \item * Reader and Director of Research, Centre for Journalism, University of Kent, UK
    \item * \href{mailto:b.cocking@kent.ac.uk}{b.cocking@kent.ac.uk}
    \item * +44 (0)1634 888987
  \end{itemize}
  \item * 2. Kerry Longman
  \begin{itemize}
    \item * East Kent College, Ashford College, UK
    \item * \href{mailto:kerry.longman@ekcgroup.ac.uk}{kerry.longman@ekcgroup.ac.uk}
  \end{itemize}
\end{itemize}

Signature

Printed from LASU ePERS on Sun, 08-Feb-2026 @ 01:05:09 pm.

\end{document}
