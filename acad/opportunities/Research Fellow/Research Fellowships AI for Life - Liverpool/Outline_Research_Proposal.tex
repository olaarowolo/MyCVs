\documentclass[a4paper,11pt]{article}

\usepackage{latexsym}
\usepackage{titlesec}
\usepackage[usenames,dvipsnames]{color}
\usepackage{verbatim}
\usepackage{hyperref}
\usepackage{fancyhdr}
\usepackage[english]{babel}
\usepackage{tabularx}
\input{glyphtounicode}

\pagestyle{fancy}
\fancyhf{} % clear all header and footer fields
\fancyfoot{}
\renewcommand{\headrulewidth}{0pt}
\renewcommand{\footrulewidth}{0pt}

% Adjust margins
\addtolength{\oddsidemargin}{-0.5in}
\addtolength{\evensidemargin}{-0.5in}
\addtolength{\textwidth}{1in}
\addtolength{\topmargin}{-.5in}
\addtolength{\textheight}{1.0in}

\urlstyle{same}

\raggedbottom
\raggedright
\setlength{\tabcolsep}{0in}

% Sections formatting
\titleformat{\section}{
  \vspace{-4pt}\scshape\raggedright\large
}{}{0em}{}[\color{black}\titlerule \vspace{-5pt}]

% Ensure that generate pdf is machine readable/ATS parsable
\pdfgentounicode=1

%-------------------------------
% Custom commands
\newcommand{\resumeItem}[1]{
  \item\small{
    {#1 \vspace{-2pt}}
  }
}

\newcommand{\resumeSubheading}[4]{
  \vspace{-2pt}\item
    \begin{tabular*}{0.97\textwidth}[t]{l@{\extracolsep{\fill}}r}
      \textbf{#1} & #2 \\
      \textit{\small#3} & \textit{\small #4} \\
    \end{tabular*}\vspace{-7pt}
}

\newcommand{\resumeSubSubheading}[2]{
    \item
    \begin{tabular*}{0.97\textwidth}{l@{\extracolsep{\fill}}r}
      \textit{\small#1} & \textit{\small #2} \\
    \end{tabular*}\vspace{-7pt}
}

\newcommand{\resumeProjectHeading}[2]{
    \item
    \begin{tabular*}{0.97\textwidth}{l@{\extracolsep{\fill}}r}
      \small#1 & #2 \\
    \end{tabular*}\vspace{-7pt}
}

\newcommand{\resumeSubItem}[1]{\resumeItem{#1}\vspace{-4pt}}

\renewcommand\labelitemii{$\vcenter{\hbox{\tiny$\bullet$}}$}

\newcommand{\resumeSubHeadingListStart}{\begin{itemize}[leftmargin=0.15in, label={}]}
\newcommand{\resumeSubHeadingListEnd}{\end{itemize}}
\newcommand{\resumeItemListStart}{\begin{itemize}}
\newcommand{\resumeItemListEnd}{\end{itemize}\vspace{-5pt}}

%-------------------------------------------
%%%%%%  OUTLINE RESEARCH PROPOSAL STARTS HERE  %%%%%%%%%%%%%%%%%%%%%%%%%%%%
\begin{document}

%----------HEADING----------
\begin{center}
    \textbf{\Huge \scshape OUTLINE RESEARCH PROPOSAL} \\
    \vspace{5pt}
    \textbf{\Large University of Liverpool Research Fellowship Scheme} \\
    \vspace{5pt}
    \textbf{\Large AI for Life: AI Systems and Structural Inequality} \\
    \vspace{5pt}
    \textbf{\Large Dr Olasunkanmi S. Arowolo} \\
    \vspace{1pt}
    \small Rochester, Kent, United Kingdom $|$ +44 (0)7487 397751 $|$ \href{mailto:oa@olaarowolo.com}{{oa@olaarowolo.com}} \\
\end{center}

% Set left alignment for all content after header
\raggedright

%-----------1. YOUR NAME, TITLE OF CURRENT POST, EMPLOYER, SCHOOL/DEPARTMENT-----------
\section{1. Your name, title of your current post, Employer, School/Department (if applicable)}
\small{Dr Olasunkanmi S. Arowolo, Post-Doctoral Researcher (Independent), Rochester, Kent, United Kingdom. No current employer affiliation; previously Lecturer at East Kent College and Assistant Lecturer at University of Kent.}

%-----------2. WHICH FRONTIER FOCUS AREA ARE YOU APPLYING AGAINST?-----------
\section{2. Which Frontier focus area are you applying against?}
\small{AI for Life: AI Systems and Structural Inequality}

%-----------3. PLEASE LIST THE UNIVERSITY DEPARTMENT YOU BELIEVE YOUR RESEARCH MOST CLOSELY ALIGNS WITH-----------
\section{3. Please list the University Department you believe your research most closely aligns with}
\small{Communication and Media, School of the Arts, University of Liverpool}

%-----------4. RESEARCH VISION (300 WORDS)-----------
\section{4. Research Vision (300 words)}
\small{This fellowship proposal aims to investigate how AI systems perpetuate or challenge structural inequalities in digital societies, with a specific focus on algorithmic journalism and media representation. Drawing on my expertise in journalism studies and media framing analysis, I will examine how AI-driven content curation and news algorithms amplify biases related to race, class, disability, and digital divides.

The research vision centers on developing a critical framework for understanding AI's role in media inequality. By combining computational social science with critical humanities perspectives, I will analyze how algorithmic systems in news platforms reinforce existing power structures while potentially offering tools for democratizing information access.

Key innovations include participatory AI design methods that involve marginalized communities in algorithm development, ensuring AI tools serve social justice rather than commercial interests. The vision extends to creating ethical AI frameworks for journalism that prioritize transparency, accountability, and inclusivity.

This work addresses the urgent need to mitigate AI's potential to deepen digital divides, particularly in welfare, employment, education, and media sectors. By focusing on algorithmic journalism, I will explore how AI can either perpetuate stereotypes or enable more equitable representation of diverse voices.

The fellowship provides the ideal platform to build interdisciplinary collaborations between communication scholars, AI researchers, and civil society organizations, ultimately contributing to more responsible AI development in media ecosystems.}

%-----------5. STRATEGIC ALIGNMENT (300 WORDS)-----------
\section{5. Strategic Alignment (300 words)}
\small{This research aligns directly with the AI for Life Frontier's focus on 'AI Systems and Structural Inequality,' hosted in the Communication and Media department. The proposal addresses how AI perpetuates or challenges inequalities in digital societies, specifically examining intersections of digital/AI divides with race, class, and disability in media contexts.

Strategically, it contributes to Liverpool's profile in AI applications by advancing critical perspectives on AI ethics in journalism. The work supports the University's commitment to inclusive AI development by emphasizing participatory methods and community involvement.

The research builds on my established expertise in media framing and journalism studies, extending this to AI contexts. My PhD research on media coverage of poverty programmes provides a foundation for analyzing algorithmic biases in news representation.

Collaboration with Liverpool's partners in civil society, trade unions, and digital rights groups will be central, leveraging their existing networks. The fellowship's tenure-track structure allows for long-term impact through curriculum development in AI ethics and media studies.

This work addresses UK Industrial Strategy priorities by promoting responsible AI in creative industries and contributes to broader societal challenges of digital inclusion. The interdisciplinary approach, combining computational methods with critical analysis, positions Liverpool as a leader in ethical AI research.

The proposal's focus on media aligns with the department's strengths in communication studies, while introducing AI innovation. It supports the University's Strategy 2031 by fostering research that addresses major societal challenges through technological advancement with human values at the center.}

%-----------6. KEY METHODOLOGICAL INNOVATIONS IN YOUR APPROACH (300 WORDS)-----------
\section{6. Key Methodological Innovations in your Approach (300 words)}
\small{The methodological approach combines computational social science with critical humanities, creating innovative frameworks for analyzing AI in media inequality.

\textbf{Algorithmic Audit Methodology:} I will develop systematic audits of news algorithms, examining how content curation systems amplify or mitigate biases. This involves quantitative analysis of algorithmic outputs using machine learning techniques to identify patterns of representation.

\textbf{Participatory AI Design:} Drawing from critical humanities, I will implement co-design methods involving marginalized communities in algorithm evaluation and redesign. This ensures AI tools reflect diverse perspectives rather than commercial priorities.

\textbf{Mixed-Methods Integration:} Combining quantitative algorithmic analysis with qualitative ethnographic studies of affected communities. This includes in-depth interviews with journalists, content creators, and audiences to understand lived experiences of AI-mediated inequality.

\textbf{Ethical AI Frameworks:} Developing transparent evaluation metrics for algorithmic fairness in journalism, including bias detection tools and accountability mechanisms.

\textbf{Interdisciplinary Collaboration:} Working with computer scientists, sociologists, and media practitioners to create hybrid methodologies that bridge technical and social perspectives.

Key deliverables include:
- Open-source algorithmic audit toolkit for media organizations
- Participatory AI design framework for inclusive algorithm development
- Case studies of AI interventions in newsrooms
- Policy recommendations for regulatory frameworks

The approach ensures reproducibility through documented methodologies and open data sharing, while prioritizing ethical considerations in AI development. This innovative blend of methods will advance both technical capabilities and societal applications of AI in media.}

%-----------7. IMPACT (250 WORDS)-----------
\section{7. Impact (250 words)}
\small{The research will maximize impact through multiple pathways, addressing short- and long-term societal benefits in AI ethics and media equity.

\textbf{Short-term Impact:} Development of practical tools for media organizations to audit algorithmic biases, enabling immediate improvements in content diversity and representation.

\textbf{Long-term Impact:} Contribution to regulatory frameworks for ethical AI in journalism, influencing policy on digital rights and media accountability.

\textbf{Collaborators and Partners:}
- Civil society organizations: Digital rights groups for participatory design
- Trade unions: Media unions for workplace AI implementation
- Academic partners: University of Kent (existing collaborations), University of Westminster
- Industry partners: News organizations for real-world testing
- International networks: African Council for Communication Education for global perspectives

Impact will be achieved through:
- Open-source toolkits for algorithmic transparency
- Policy briefs for UK government and regulators
- Training programs for journalists on AI ethics
- Public engagement through media articles and conferences
- Knowledge exchange with industry partners

The work directly supports Liverpool's partnerships with organizations addressing digital inequalities, contributing to the University's reputation in responsible AI research. By focusing on media, it addresses fundamental questions of representation and power in digital societies, with potential to influence global standards for AI in creative industries.

Expected outcomes include improved media diversity, enhanced public trust in AI-mediated information, and frameworks for inclusive AI development that can be applied beyond journalism to other sectors.}

%-----------8. LEADERSHIP (250 WORDS)-----------
\section{8. Leadership (250 words)}
\small{This fellowship represents the optimal opportunity to transition from my journalism expertise to leading AI research in media inequality. My PhD completion (no corrections) and British Academy Global Talent endorsement demonstrate research excellence and international recognition.

\textbf{Current Leadership Experience:}
- Founder/CEO of Afriscribe, managing research dissemination platform
- Director of ResearchAfrica, leading academic network
- Editorial roles: Blind peer reviewer for Q2 journal and Taylor \& Francis
- Teaching leadership: Developed curricula for 950+ students across UK and Nigeria

\textbf{Fellowship Benefits:}
- Structured career progression plan for developing research leadership
- Access to Liverpool's interdisciplinary networks in AI and media
- Mentorship and training in advanced AI methodologies
- Platform to build research group in algorithmic journalism ethics

\textbf{Leadership Development Plan:}
- Year 1: Establish research programme, secure funding, build collaborations
- Year 2-3: Expand team, develop impact outputs, supervise early-career researchers
- Year 4-5: Lead major projects, contribute to curriculum development

\textbf{Research Group Vision:}
Build interdisciplinary team combining journalism scholars, AI specialists, and community partners to address AI inequality in media. Focus on mentoring diverse early-career researchers and fostering inclusive research practices.

The fellowship's tenure-track structure provides stability to develop long-term leadership in this emerging field. My commitment to EDI, demonstrated through mentoring underrepresented students and research on marginalized voices, aligns with Liverpool's values.

This opportunity will position me as a leader in responsible AI for media, contributing to both academic excellence and societal impact.}

\end{document}
